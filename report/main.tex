\documentclass{article}


% if you need to pass options to natbib, use, e.g.:
%     \PassOptionsToPackage{numbers, compress}{natbib}
% before loading neurips_2023


% ready for submission
\usepackage[final]{neurips_2023}


% to compile a preprint version, e.g., for submission to arXiv, add add the
% [preprint] option:
%     \usepackage[preprint]{neurips_2023}


% to compile a camera-ready version, add the [final] option, e.g.:
%     \usepackage[final]{neurips_2023}


% to avoid loading the natbib package, add option nonatbib:
%    \usepackage[nonatbib]{neurips_2023}


\usepackage[utf8]{inputenc} % allow utf-8 input
\usepackage[T1]{fontenc}    % use 8-bit T1 fonts
\usepackage{hyperref}       % hyperlinks
\usepackage{url}            % simple URL typesetting
\usepackage{booktabs}       % professional-quality tables
\usepackage{amsfonts}       % blackboard math symbols
\usepackage{nicefrac}       % compact symbols for 1/2, etc.
\usepackage{microtype}      % microtypography
\usepackage{xcolor}         % colors


% OWN PACKAGES AND SETTINGS

\usepackage{amsmath,amsthm,amssymb} % for the common "math symbols"
\usepackage{mathrsfs}
\usepackage{mathtools}
\usepackage{caption}
\usepackage{subcaption}
\usepackage{wrapfig}



% STRUCTURE
\newtheorem{theorem}{Theorem}


% CUSTOM COMMANDS
\newcommand{\set}[1]{\left\{#1\right\}}
\newcommand{\multiset}[1]{\left\{\!\!\left\{#1\right\}\!\!\right\}}
\newcommand{\iter}[1]{^{(#1)}}
\newcommand{\wl}{\texttt{wl}}
\newcommand{\wledge}{\texttt{wl-ed}}
\newcommand{\lwl}{\texttt{lwl}}
\newcommand{\norm}[1]{\left\lVert#1\right\rVert}
\newcommand{\upd}{\texttt{UPD}}
\newcommand{\agg}{\texttt{AGG}}
\newcommand{\dec}{\xi}
\newcommand{\hash}{\tau}
\newcommand{\nbh}{\mathcal{N}}
\newcommand{\bs}[1]{\boldsymbol{#1}}
\newcommand{\feat}{\bs{h}}

\newcommand{\mca}{\mathcal{A}}
\newcommand{\mcb}{\mathcal{B}}
\newcommand{\mcc}{\mathcal{C}}
\newcommand{\mcd}{\mathcal{D}}
\newcommand{\mce}{\mathcal{E}}
\newcommand{\mck}{\mathcal{K}}
\newcommand{\mcl}{\mathcal{L}}
\newcommand{\mcm}{\mathcal{M}}
\newcommand{\mcn}{\mathcal{N}}
\newcommand{\mcq}{\mathcal{Q}}
\newcommand{\mcs}{\mathcal{S}}
\newcommand{\mct}{\mathcal{T}}
\newcommand{\mcv}{\mathcal{V}}

\newcommand{\mbb}{\mathbb{B}}
\newcommand{\mbe}{\mathbb{E}}
\newcommand{\mbi}{\mathbb{I}}
\newcommand{\mbk}{\mathbb{K}}
\newcommand{\mbn}{\mathbb{N}}
\newcommand{\mbr}{\mathbb{R}}
\newcommand{\mbz}{\mathbb{Z}}

\newcommand{\msk}{\mathscr{K}}
\newcommand{\msl}{\mathscr{L}}
\newcommand{\msw}{\mathscr{W}}



\title{Expressiveness of Line Graph Neural Networks}
\author{%
    1083152
}


\begin{document}
\maketitle


\begin{abstract}

\end{abstract}


\section{Introduction}
% Start with graph structured data - ubiquitous in real-world applications, such as social networks, biological networks, recommendation systems, material science, traffic networks, etc.

% Graph neural networks

% Line graph neural networks - especially natural for the task of link prediction

% Expressive power

% State research question

% Summarize contributions



\section{Related Work}

\paragraph{Graph Neural Networks}
% focus on isomorphism and permutation invariance/equivariance


\paragraph{Expressiveness}
% start with expressive power of neural networks in general
% then move to graph neural networks
% state both graph/node distinguishability and function approximation as ways to characterize expressiveness
% 

\paragraph{Line Graph Neural Networks}



\section{Background}    \label{sec:background}
% Summary of notations?

\paragraph{Graphs}
% + Line graphs
% + Graph Colourings
% + Node colourings of line graphs

\paragraph{Graph Neural Networks}
% Framework with update/aggregate functions

\paragraph{Graph Colourings and }




\section{Proposed Approach}




\section{Theory}
% TODO: write something about edge colour initializations

% define lwl and wledge
% prove that lwl is less powerful than wledge
% prove that wl is equally powerful as wledge
% connection to k-wl

In this section we study the expressive power of Line Graph MPNNs (LG-MPNNs), which we define as MPNNs (outlined in Section \ref{sec:background}) applied to the line graph of a graph $G$.
The \emph{Line GNNs} proposed in \cite{cai2021line} fall under this framework.
We do this from a theoretical graph isomorphism testing perspective, by comparing them to variations of the WL-test in terms of their power to distinguish non-isomorphic graphs.
Because Line Graph MPNNs compute edge features of the original graph rather than node features, we devise two new variations of the standard WL-test that are also tailored to computing edge colourings. 
Afterwards, we relate the expressive power of both these variations to each other, to Line Graph MPNNs, and to the standard WL-test.


\subsection{Two variants of the Weisfeiler-Leman test for edge colouring}

The first variant of the WL-test we introduce is the \emph{Line Weisfeiler-Leman Test}, denoted $\lwl$, which is a direct application of the standard WL-algorithm to the line graph. Concretely, we define the $\lwl$ algorithm as follows:
\begin{itemize}
    \item Initialize the colour of each node $uv \in V(L(G))$ to $\lwl\iter{0}(G)(uv) = \dec(\set{c(u),c(v)})$, where $\dec$ is an injective function from $\mcc^2$ to $\mcc$. Note that such a function exists, because we can choose $\mcc$ to be countable (e.g. $\mcc = \mbn$), in which case the cardinalities of $\mcc^2$ and $\mcc$ are equal.
    \item Iteratively update the colour of each node $uv \in V(L(G))$ as follows:
    \begin{equation}
        \lwl\iter{t+1}(G)(uv) = \hash\left(\lwl\iter{t+1}(G)(uv), \multiset{\lwl\iter{t}(G)(xy) \mid xy \in \nbh(uv)}\right)
    \end{equation}
    where $\hash$ is an injective map from $\mcc\times\mbn^\mcc$ to $\mcc$. We remind the reader that $\nbh(uv)$ denotes the neighbourhood of $uv$ in $L(G)$, i.e. $\nbh(uv) = \set{d_uv \mid d_u \in \nbh_G(u)} \cup \set{ud_v \mid d_v \in \nbh_G(v)}$.
\end{itemize}
The design choice of initializing the colour of $uv$ to $\dec(\set{c(u),c(v)})$ is a natural and general way to encode the initial node colours of $u$ and $v$ into the edge colour of $uv$ in a permutation-invariant way. This permutation invariance is important, as $uv=vu$ in the line graph.


The second variant we introduce is the \emph{Weisfeiler-Leman Test with Edge Decoding}, denoted $\wledge$, which enhances the standard WL-algorithm with an injective decoder $\dec: \mcc^2\rightarrow\mcc$ to transform the node colours into edge colours. 
This decoder need not be the same as the one used in $\lwl$, but we overloaded the notation because the only thing that matters for the expressivity is that it is injective.
Concretely, it applies the WL-algorithm to the original (coloured) graph $G$, computing a node colouring $\wl^t(G): V(G) \rightarrow \mcc$ at each iteration $t$. Then the edge colouring at iteration $t$ is computed as $\wledge\iter{t}(G)(uv) = \dec(\set{\wl^t(G)(u),\wl^t(G)(v)})$.


In what follows, we will first prove that LG-MPNNs are at most as expressive as the $\lwl$-test and that there exist LG-MPNNs that are equally expressive as $\lwl$, where we measure expressiveness in terms of ability to distinguish non-isomorphic graphs.
Afterwards, we will show that $\lwl$ is less expressive than $\wledge$ and that $\wledge$ is equally expressive as the standard WL-test.

\begin{theorem}
    LG-MPNNs are at most as expressive as the $\lwl$-test. 
\end{theorem}

\begin{proof}
    Previous work on the expressivity of standard MPNNs \cite{morris2019weisfeiler} has shown that, for any MPNN with $T$ layers, and for all labelled graphs $G$ and $t\in\set{0,\dots,T}$, the WL-test's node colouring $\wl\iter{t}(G)$ is a refinement of the MPNN's node features $\feat\iter{t}(G)$ at iteration $t$:
    \begin{equation}
        \wl\iter{t}(G) \preceq \feat\iter{t}(G)
    \end{equation}

    Applying this theorem to the line graph of $G$ with initial node colourings defined by $l: V(L(G)) \rightarrow \mcc: uv \mapsto \dec(\set{c(u),c(v)})$, one obtains that for any LG-MPNN with $T$ layers, and for all labelled graphs $G$ and $t\in\set{0,\dots,T}$:
    \begin{equation}    \label{eq:lwl-refinement-of-lg-mpnn}
        \lwl\iter{t}(G) \preceq \feat\iter{t}(G)
    \end{equation}

    In particular, if the LWL-test cannot distinguish between two non-isomorphic graphs $G_1$ and $G_2$, then that implies
    \begin{equation}
        \set{\lwl\iter{T}(G_1)(u) \mid u\in V(G_1)} = \set{\lwl\iter{T}(G_2) \mid u\in V(G_2)}
    \end{equation}
    Applying (\ref{eq:lwl-refinement-of-lg-mpnn}) to the joint graph $G=G_1\cup G_2$ yields 
    \begin{equation}
        \set{\feat\iter{T}(u) \mid u\in V(G_1)} = \set{\feat\iter{T}(u) \mid u\in V(G_2)}
    \end{equation}
    from which follows that the LG-MPNN cannot distinguish between $G_1$ and $G_2$ either.
\end{proof}

\begin{theorem}
    There exists an LG-MPNNs that are equally expressive as the $\lwl$-test.
\end{theorem}

\begin{proof}
    \cite{xu2018powerful} proved that for any finite set of graphs $\set{G_1, \dots, G_N}$ that are pairwise distinguishable by the WL-test, there exists an MPNN that can distinguish between them.
    % For this they introduced the \emph{Graph Isomorphism Network} architecture.

    Now consider any set $\set{G_1, \dots, G_N}$ that is pairwise distinguishable by the LWL-test. This is equivalent to saying $\set{L(G_1), \dots, L(G_N)}$ is pairwise distinguishable by the WL-test. From the previous result now follows that there exists an MPNN that can distinguish the line graphs, i.e. that there exists an LG-MPNN that can distinguish between the original graphs.
\end{proof}

\begin{theorem}
    The $\lwl$-test is strictly less expressive than the $\wledge$-test.
\end{theorem}


\newlength{\WLOGarrowwidth}
\settowidth{\WLOGarrowwidth}{$\stackrel{\text{WLOG}}{\Rightarrow}$}
\newcommand{\RightarrowAsWideAsWLOGArrow}{\makebox[\WLOGarrowwidth][c]{$\Rightarrow$}}
\begin{proof}
    The proof of this theorem consists of two parts. First, we show that for any graph $G$, $\wledge\iter{t}(G) \preceq \lwl\iter{t}(G)$ for all $t\in\mbn$. Second, we show that there exist graphs $G_1$ and $G_2$ that are distinguishable by $\wledge$ but not by the $\lwl$.

    For the first part, we prove the following by induction on $t$. For any graph $G$ and $t\in\mbn$:
    \begin{equation}    \label{eq:wledge-refinement-of-lwl-induction-hypothesis}
        \wledge\iter{t}(G) \preceq \lwl\iter{t}(G)
    \end{equation}
    Fix the graph $G$ and shorten the notation $\wl\iter{t} \coloneq \wl\iter{t}(G)$ for clarity (and analogous for $\lwl$ and $\wledge$). The base case $t=0$ is trivially true, because the initial edge colourings of $\wledge$ are the same as those of $\lwl$. For the induction step, assume (\ref{eq:wledge-refinement-of-lwl-induction-hypothesis}) holds for $t$. Then take any $uv, xy \in V(L(G))$ for which $\wledge\iter{t+1}(uv) = \wledge\iter{t+1}(xy)$. Due to the injectivity of $\hash$ and the decoders $\dec$, the following implications hold:
    \begin{equation}
        \begin{split}
            &\wledge\iter{t+1}(uv) = \wledge\iter{t+1}(xy)
            \\
            &\RightarrowAsWideAsWLOGArrow
            \set{\wl\iter{t+1}(u),\wl\iter{t+1}(v)} = \set{\wl\iter{t+1}(x),\wl\iter{t+1}(y)}
            \\
            &\makebox[\WLOGarrowwidth][c]{$\stackrel{\text{WLOG}}{\Rightarrow}$}
            \enspace \wl\iter{t+1}(u) = \wl\iter{t+1}(x) \wedge \wl\iter{t+1}(v) = \wl\iter{t+1}(y)
            \\
            &\RightarrowAsWideAsWLOGArrow
            \begin{cases}
                \wl\iter{t}(u) = \wl\iter{t}(x) \wedge \multiset{\wl\iter{t}(d_u) \mid d_u \in \nbh(u)} = \multiset{\wl\iter{t}(d_x) \mid d_x \in \nbh(x)} \\
                \wl\iter{t}(v) = \wl\iter{t}(y) \wedge \multiset{\wl\iter{t}(d_v) \mid d_v \in \nbh(v)} = \multiset{\wl\iter{t}(d_y) \mid d_y \in \nbh(y)}
            \end{cases}
            \\
            &\RightarrowAsWideAsWLOGArrow 
            \begin{cases}
                \wledge\iter{t}(uv) = \wledge\iter{t}(xy) \\
                \multiset{\wledge\iter{t}(ud_u) \mid d_u \in \nbh(u)} = \multiset{\wledge\iter{t}(xd_x) \mid d_x \in \nbh(x)} \\
                \multiset{\wledge\iter{t}(vd_v) \mid d_v \in \nbh(v)} = \multiset{\wledge\iter{t}(yd_y) \mid d_y \in \nbh(y)}
            \end{cases}
            \\
            &\makebox[\WLOGarrowwidth][c]{$\stackrel{(\ref{eq:wledge-refinement-of-lwl-induction-hypothesis})}{\Rightarrow}$}
            \begin{cases}
                \lwl\iter{t}(uv) = \lwl\iter{t}(xy) \\
                \multiset{\lwl\iter{t}(ud_u) \mid d_u \in \nbh(u)} = \multiset{\lwl\iter{t}(xd_x) \mid d_x \in \nbh(x)} \\
                \multiset{\lwl\iter{t}(vd_v) \mid d_v \in \nbh(v)} = \multiset{\lwl\iter{t}(yd_y) \mid d_y \in \nbh(y)}
            \end{cases}
            \\
            &\RightarrowAsWideAsWLOGArrow
            \lwl\iter{t+1}(uv) = \lwl\iter{t+1}(xy)
        \end{split}
    \end{equation}
    Because this holds for all $uv,xy\in V(L(G))$, we have shown that $\wledge\iter{t+1}(G) \preceq \lwl\iter{t+1}(G)$, which concludes the induction step.
    
    % TODO: draw conclusions for `graph distinguishability'

    For the second part, consider the following two graphs $G_1$ and $G_2$:
    % TODO: draw the graphs
\end{proof}


\section{Experiments}



\section{Outlook and Conclusion}

% Extend to edge-attributed graphs
% Extend to directed graphs
% Extend to higher-order lemmas
% Extend to knowledge graphs


\bibliographystyle{unsrt}
\bibliography{refs}


\end{document}